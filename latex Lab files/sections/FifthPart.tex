\section{Fifth Member}
This is the section dedicated to one of the team members, and it should be written individually . It can include a range of things; first subsection is a space for you to point out the strengths and weaknesses of the module, including complaints about the module coordinator Max Wilson. The second section should have a selfie image with Max! The last part of it is the most important one. You will need to write a paragraph about what you have learned in this module. You can write it in \textbf{Bold} if you want or you can use other fonts. 

Please do not forget:
\begin{itemize}
	\item First paragraph shoud have your cmments about the module
	\item Second one, a selfie img with Max
	\item Last one, what you learned in this module.
\end{itemize}

\subsection{Comments about the module}
The module is well taught, since Max has done a lot to organise information and make it clear how different parts are related to each other. The testing lab was really boring which has taught us why so many companies ignore testing and do it badly - so I guess there's a lesson in there somewhere. Other parts are more interesting, and Max has made sure there's no lack of content as he has supplied a lot of reading which is helpful. Complaints about Max:
\begin{itemize}
	\item Thinks he's down with the kids.
	\item Doesn't get hammered enough during lectures.
	\item Isn't teaching me astronomy.
	\item Made us do that testing lab.
	\item Why didn't I apply to an astronomy degree?
	\item Too sly about optional content.
	\item Only did the selfie thing because this morning he realised he's not been in enough selfies.
	\item Probably can't tell me which universities offer astronomy degrees.
\end{itemize}

\subsection{Selfie with Max}

To include an image, you will need to remove the comments from the code below, place an image in the main folder, and do not forget to put the name of the image instead of ImgName. 

\begin{figure}[h]
\caption{Selfie with Max}
\centering
\includegraphics[width=0.5\textwidth]{selfieRufus.jpg}
\label{fig:selfie}
\end{figure}

You can then use the label of the figure to reference it later with the command ${\backslash}ref$. you can comment out the next line to see an example of how it works.

My selfie with Max is in  Figure~\ref{fig:selfie}.

\subsection{What I have learned in this module}
In this module I learnt what my Dads job is and that perhaps it's more interesting than I thought. Specifically, I learnt about development strategies and the necessary steps involved in any software project, as well as why companies might fail in these steps and how to avoid failing. Essentially it all comes down to having experienced people in charge , who can keep upper management happy and who understand that more coders doesn't equal faster work. I also learnt how important steps in software development get executed - such as requirements gathering and testing, how these steps are very much dependent on one another, and how even if a module is called "software development" it doesn't necessarily involve developing software.

However the best way to summarise what I learned in this module would be with the place holder text that was here originally.
"This is some random text."


